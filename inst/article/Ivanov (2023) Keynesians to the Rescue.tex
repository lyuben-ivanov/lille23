% AER-Article.tex for AEA last revised 22 June 2011
%\documentclass[AER, draftmode]{AEA}
%\documentclass[AER, reviewmode]{AEA}



\documentclass[AER]{AEA}


\usepackage{natbib}
\usepackage[T1,T2A]{fontenc}
\usepackage[utf8]{inputenc}
\usepackage[english]{babel}
\usepackage{microtype}
\usepackage{parskip}
\usepackage{amsmath}
\usepackage{tikz}
\usepackage{lipsum}


\draftSpacing{1.5}

\begin{document}


\title{Keynesians to the Rescue: Unprecedented Policy Responses Towards Unprecedented Macroeconomic Shocks (Evidence from Three Natural Experiments)}
\shortTitle{Draft}
\author{Lyuben Ivanov\thanks{Faculty of Economics and Business at Sofia University St. Kliment Ohridski\dq, Sofia, Bulgaria (e-mail: l.ivanov@feb.uni-sofia.bg)}}
\date{2023}
\pubMonth{Nov}
\pubYear{2023}
\pubVolume{}
\pubIssue{}
\JEL{{\normalfont E32, E44, E58, E62, F3, G1, G4}}
\Keywords{global financial and economic crisis, COVID-19 pandemic}

\begin{abstract}
The Great Recession of the 2000s and the Covid Crisis of the 2020s had different causes but elicited very similar government fiscal responses in some of the largest global economies. In both cases, the governments of the United States and the Euro Area countries enacted unprecedent fiscal stimuluses on the background of unprecedented monetary expansion by the respective central banks. Despite the different causes, in both historical episodes the world economy’s industrial production, employment and equity markets recovered much faster compared to another crisis that started with a similar magnitude but developed on the background of much more muted policy responses –- the Great Depression of the 1930s.  The current study compares the dynamics of the economic measures listed above during the Great Recession and the Covid Crisis with their dynamics during the Great Depression. It also compares the magnitude of the policy responses for the historical episodes listed above. A qualitative analysis is performed to provide background and shed light on the economic and policy developments that have resulted in the different dynamics of the above measures during the different historical episodes. 
\end{abstract}

\maketitle

\section{Descriptive Analysis of Three Major Economic Crisis}

\subsection*{A. \quad Dynamics of Industrial Production During the Great Depression, the Great Recession and the COVID-19 Crisis} 

%\begin{figure}[h!]
%     \centering
%     \input{figure_1.tex}
%     \caption{Възстановяване на финансовите пазари от два макроикономически шока}
%     \label{fig:indices}
%     \begin{figurenotes}[Бележки]
% Трите индекса са преизчислени за да приемат стойност 0 в началото на всеки от двата времеви периода. В резултат на преизчислението, стойностите на всеки от индексите в даден момент показват процентната промяна спрямо стойността на индекса в началото на периода. На графиките S\&P 500 е обозначен като САЩ, EURO STOXX 50 е обозначен като ЕЗ, а SOFIX е обозначен като БГ. 
%     \end{figurenotes}
%%     \begin{figurenotes}[Източници]
%%      www.investor.bg; Изчисления на автора. 
%%     \end{figurenotes}
%\end{figure}  
%
%Графика \ref{fig:indices} ни показва динамиката на SOFIX, EURO STOXX 50 и S\&P 500 по време на световната финансова и икономическа криза и по време на пандемията от Ковид-19. 


\newpage

\bibliographystyle{aea}
\bibliography{bib}



\end{document}

